\documentclass[10pt, twocolumn]{article}

\usepackage[pdftex]{graphicx}
\usepackage{fullpage}
\setlength{\parindent}{0pt}
\setlength{\parskip}{\baselineskip}



\title{Lab Rotation Proposal \\ Modeling Bursting in Somatosensory
Cortex}
\author{Valentin H\"anel \\ \small Bernstein Center for
Computational Neuroscience Berlin}    

\date{\today}

\pagestyle{empty}
\begin{document} \maketitle

\section{Overview}

Going from extracellularly recorded spike trains to spiking neuronal
models has been a topic of research during the last few years. One
example is a Poisson stochastic encoding models  \cite{ BerryMeister, 
DayanAbbott}. The simple models may be fit to a large amount of
experimental data, but they do not contain biophysical parameters such
as conductance or membrane potential. An alternative technique based
on the leaky integrate and fire model \cite{PaninskiPillowSimoncelli}
has been shown to describe experimental data better and its parameters
have a straightforward biophysical interpretation. The drawback,
however, is that such models are more complex and thus fitting of the
parameters will be somewhat more involved. 

\section{Goals}

The proposed goals of the lab rotation are as follows:
\begin{enumerate}

    \item Implement and compare selected models of spiking neural
        activity (Poisson, Cascade, Integrate and Fire)
    \item Implement and compare methods to fit parameters of the
        models to experimental spike trains.
    \item Evaluate the models with respect to spike bursts recorded
        from somatosensory cortex of behaving macaque monkeys.

\end{enumerate}

\section{Time Requirements}

Due to additional constraints imposed by courses on advanced topics
and a job as student assistant, a total of approximately 3 working
days per week will be spent on the Lab rotation.

Additional reading of textbooks and journal papers and preparation of
the proposal took 3 weeks. From 10.11.08 to the 28.11.08.

The official start of the lab rotation according to the TU Examination
office was 24.11.08.

Each proposed goal should take around 3 weeks, and therefore the
proposed end of the lab rotation is end of January.

\bibliography{proposal}{ \bibliographystyle{abbrv} }

\end{document}

