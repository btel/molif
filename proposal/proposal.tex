\documentclass[10pt, twocolumn]{article}

\usepackage[pdftex]{graphicx}
\usepackage{fullpage}



\title{Lab Rotation Proposal}
\author{Valentin H\"anel \\ \small Bernstein Center for
Computational Neuroscience Berlin}    

\date{\today}

\begin{document} \maketitle

\section{Overview}

Going from extracellularly recorded spike trains to spiking neuronal
models has been a topic of research during the last few years. One
example is a poisson spike generator as described in
\cite{DayanAbbott} or \cite{BerryMeister}. However this simple model
does not allow for an estimation of the membrane potential. An
alternative technique based around the leaky integrate and fire mode
has been proposed in \cite{PaninskiPillowSimoncelli}. This  uses a maximum
likelihood based technique and due to the nature of the model does
allow for an estimation of membrane potential.
Furthermore in \cite{PaninskiHaithSzirtes} a slightly better model is
proposed.WHY IS IT BETTER.

The first proposed goal of the lab rotation is to implement the
techniques described for the integrate and fire model,
\cite{PaninskiHaithSzirtes} and use it fit the experimental data
recorded by Bartosz. INSERT ADDITIONAL DETAIL 

The second proposed goal of the lab rotation is to lay the foundations
for a modeling framework written in python. This should allow the
neuroscientist to input a model and its objective function, and have the framework 
take care of fitting and assessment of the quality of the fit.

The third proposed goal of the lab rotation is to implement the simple
poisson spike generator within the proposed framework and then compare
how appropriate both models are for the experimental data gathered.
Cross examination of models, where the simulation output of one model
is the input for the fitting of another, is an option. 

\section{Time Requirements}

Due to additional constraints imposed by courses on advanced topics
and a job as student assistant, a total of approximately 3 working
days per week will be spent on the Lab rotation.

Additional reading of textbooks and journal papers and preparation of
the proposal took 3 weeks. From 10.11.08 to the 28.11.08.

The official start of the lab rotation according to the TU Examination
office was 24.11.08.

Each proposed goal should take around 3 weeks, and therefore the
proposed end of the lab rotation is end of January.

\bibliography{proposal}{ \bibliographystyle{plain} }

\end{document}

