\documentclass[10pt]{article}
\usepackage[pdftex]{graphicx}
\usepackage{amssymb, amsmath}
\usepackage{fullpage}
\setlength{\parindent}{0pt}
\setlength{\parskip}{\baselineskip}



\begin{document} 

\section{Equations}

We need to solve the following fokker-planck diffusion equation
numerically.

\begin{equation}
    \frac{\partial P(V,t)}{\partial t} =
    \frac{1}{2} \frac{\partial^2 P(V,t) } {\partial V^2} +
    g\frac{\partial[(V-V_{rest})]P(V,t)}{\partial V}
\end{equation}

Given the boundary conditions:

We can rewrite this as:

\begin{equation}
    \frac{1}{2} \frac{\partial^2 P(V,t) } {\partial V^2} +
    g(V-V_{rest})\frac{\partial P(V,t)}{\partial V} +
    gP(V,t) -
    \frac{\partial P(V,t)}{\partial t} = 
    0
\end{equation}

Since we know that $ V-V_{rest} = 0 $

Next we discretize time and potential. Time is discretized  into $U$
intervals of length $u$ and indexed by: $ 0,1, \dots \tau $ Potential
is discretized into $W$ intervals of length $w$ and indexed by $ 0,1,
\dots \nu $. We adopt the notation that $P_{\nu,\tau} = P(\nu,\tau)$ 

Using the crank-nicolson scheme we  may now rewrite the derivatives as:

\begin{align*}

%\begin{equation}
    P &= \frac{P_{\nu,\tau} + P_{\nu,\tau + 1}}{2} \\
%\end{equation}


%\begin{equation}
    \frac{\partial P}{\partial t} &= \frac{P_{\nu,\tau +1 } - P_{\nu,\tau}}{u} 
%\end{equation}


%\begin{equation}
    \frac{\partial P}{\partial V} &= 
    \frac{P_{\nu +1,\tau } + P_{\nu +1,\tau +1 } -
    P_{\nu - 1,\tau } P_{\nu -1,\tau +1}} 
    {4w}
%\end{equation}


%\begin{equation}
    \frac{\partial^2 P}{\partial V^2} &= 
    \frac{P_{\nu+1,\tau} - 2 P_{\nu,\tau} + P_{\nu-1,\tau} +
    P_{\nu+1,\tau+1} - 2 P_{\nu,\tau+1} + P_{\nu-1,\tau+1}}
    {2w^2} 
%\end{equation}

\end{align*}



\end{document}
